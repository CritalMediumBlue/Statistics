\documentclass[12pt]{article}

% Packages for mathematics
\usepackage{amsmath, amssymb}

% For including graphics
\usepackage{graphicx}
 
% For better control over margins
\usepackage[margin=1in]{geometry}
 
% For indicator function
\usepackage{bbm} 

 % For alphabetical enumeration
\usepackage{enumitem}


% Define custom environment for solutions
\newenvironment{solution}
  {\par\noindent\textbf{}}
  {\par\bigskip}

\title{Introduction to statistics}
\author{Prof. Dr. Christoph Richard, Leonie Wicht, Anna Vandebosch and Theresa Schmid}

\date{\today}

\begin{document}

\maketitle

% Question 1
\section*{T1. Empirical Cumulative Distribution Function}

Let $x_1, x_2, \ldots, x_n$ be a data sequance with empirical cumulative distribution function $F_n(t)$ and relative interval frecuencies $h_n(I)$, i.e
\[
F_n: \mathbb{R} \rightarrow [0,1], t \mapsto h_n((-\infty,t])=\frac{1}{n}\sum_{i=1}^n \mathbbm{1}_{(-\infty,t]}(x_i)
\]


Show that for any real numbers $a < b$ we have:

\begin{align*}
h_n((a,b]) &= F_n(b) - F_n(a)\\
h_n(\{a\}) &= F_n(a)-F_n(a-)\\
h_n([a,b]) &= F_n(b)-F_n(a)+h_n(\{a\})\\
h_n([a,b)) &= F_n(b-)-F_n(a-)\\
h_n((a,\infty)) &= 1-F_n(a)
\end{align*}

Here $F(t-):=\lim_{x\uparrow t}F(x)$ denotes the limit from the left. If you manipulate the indicator functions, give proof of any of these rules.


% Solution 1
\subsection*{Solution T1}


To prove these rules, we first have to notice the following manipulations of the indicator function:
\begin{enumerate}[label=\textbf{\alph*.   }]
    \item $ \mathbbm{1}_{(a,b]}(x) = \mathbbm{1}_{(-\infty,b]}(x)-\mathbbm{1}_{(-\infty,a]}(x) $

    \textit{Proof:} 
    
    \begin{align*}
    \mathbbm{1}_{(a,b]}(x) &= \begin{cases}
        0 & \text{if } x\notin (a,b]\\
        1 & \text{if } x\in (a,b]
    \end{cases}\\
    \mathbbm{1}_{(-\infty,b]}(x)-\mathbbm{1}_{(-\infty,a]}(x)&=\begin{cases}
        0 & \text{if } x\notin (-\infty,b] \land x\notin (-\infty,a] \quad \lor \quad x\in (-\infty,b] \cap (-\infty,a] \\       
        1 & \text{if } x\in (-\infty,b]\land x\notin (-\infty,a]\\
        -1 & \text{if } x\notin (-\infty,b]\land x\in (-\infty,a]
    \end{cases}\\
    \text{Since $a<b$, we have}\\
    \mathbbm{1}_{(-\infty,b]}(x)-\mathbbm{1}_{(-\infty,a]}(x)&=\begin{cases}
        0 & \text{if } x\notin (-\infty,b]  \lor x\in (-\infty,a] \\       
        1 & \text{if } x\in (-\infty,b]\land x\notin (-\infty,a]
    \end{cases}\\
    &=\begin{cases}
        0 & \text{if } x\notin (a,b]\\
        1 & \text{if } x\in (a,b]
    \end{cases}
\end{align*}


\item $ \mathbbm{1}_{\{a\}}(x)=\mathbbm{1}_{(-\infty,a]}(x)-\lim_{t \uparrow a}\mathbbm{1}_{(-\infty,t]}(x) $

\textit{Proof:}
\begin{align*}
    \mathbbm{1}_{\{a\}}(x)&=\begin{cases}
        0 & \text{if } x\notin \{a\}\\
        1 & \text{if } x\in \{a\}
    \end{cases}\\
                        \mathbbm{1}_{(-\infty,a]}(x)-\lim_{t \uparrow a}\mathbbm{1}_{(-\infty,t]}(x)&=
    \lim_{t \uparrow a}(\mathbbm{1}_{(-\infty,a]}(x)-                   \mathbbm{1}_{(-\infty,t]}(x))
\end{align*}

Because of what we already showed in \textbf{a}, we have that $ \mathbbm{1}_{(-\infty,a]}(x)-\mathbbm{1}_{(-\infty,t]}(x)=\mathbbm{1}_{(t,a]}(x) $, so we can continue as follows:

\begin{align*}
    \lim_{t \uparrow a}(\mathbbm{1}_{(-\infty,a]}(x)-\mathbbm{1}_{(-\infty,t]}(x))&=\lim_{t \uparrow a}\mathbbm{1}_{(t,a]}(x)\\
    &=\begin{cases}
        0 & \text{if } x\notin \{a\}\\
        1 & \text{if } x\in \{a\}
    \end{cases}\\
    &=\mathbbm{1}_{\{a\}}(x)
\end{align*}


\end{enumerate} 

    



     Now that we have shown these two manipulations (\textbf{a} and \textbf{b}), we can start proving the rules.
    \begin{enumerate}[label= \textbf{\arabic*.   }]

    \item  $ h_n((a,b]) = F_n(b) - F_n(a) $
    \begin{align*}
        h_n((a,b]) &= \frac{1}{n}\sum_{i=1}^n \mathbbm{1}_{(a,b]}(x_i)\\
        &= \frac{1}{n}\sum_{i=1}^n \mathbbm{1}_{(-\infty,b]}(x_i)-\mathbbm{1}_{(-\infty,a]}(x_i)\\
        &= \frac{1}{n}\sum_{i=1}^n \mathbbm{1}_{(-\infty,b]}(x_i)-\frac{1}{n}\sum_{i=1}^n \mathbbm{1}_{(-\infty,a]}(x_i)\\
        &= F_n(b) - F_n(a)
        \end{align*}


        \item  $ h_n(\{a\}) = F_n(a)-F_n(a-) $
    \begin{align*}
        h_n(\{a\}) &= \frac{1}{n}\sum_{i=1}^n \mathbbm{1}_{\{a\}}(x_i)\\
        &= \frac{1}{n}\sum_{i=1}^n \mathbbm{1}_{(-\infty,a]}(x_i)-\lim_{t\uparrow a}\mathbbm{1}_{(-\infty,t]}(x_i)\\
        &= F_n(a)-F_n(a-)
        \end{align*}
    
        \item $ h_n([a,b]) = F_n(b)-F_n(a)+h_n(\{a\}) $
    \begin{align*}
        h_n([a,b]) &= \frac{1}{n}\sum_{i=1}^n \mathbbm{1}_{[a,b]}(x_i)\\
        &= \frac{1}{n}\sum_{i=1}^n \mathbbm{1}_{(-\infty,b]}(x_i)-\mathbbm{1}_{(-\infty,a)}(x_i)\\
        &= \frac{1}{n}\sum_{i=1}^n \mathbbm{1}_{(-\infty,b]}(x_i)-\frac{1}{n}\sum_{i=1}^n \mathbbm{1}_{(-\infty,a]}(x_i)+\frac{1}{n}\sum_{i=1}^n \mathbbm{1}_{\{a\}}(x_i)\\
        &= F_n(b)-F_n(a)+h_n(\{a\})
        \end{align*}
    
        \item $ h_n([a,b)) = F_n(b-)-F_n(a-) $

    \begin{align*}
        h_n([a,b)) &= \frac{1}{n}\sum_{i=1}^n \mathbbm{1}_{[a,b)}(x_i)\\
        &= \frac{1}{n}\sum_{i=1}^n \mathbbm{1}_{(-\infty,b)}(x_i)-\mathbbm{1}_{(-\infty,a)}(x_i)\\
        &= \frac{1}{n} \sum_{i=1}^n \lim_{t\uparrow b}\mathbbm{1}_{(-\infty,t]}(x_i)- \lim_{t\uparrow a}\mathbbm{1}_{(-\infty,t]}(x_i)\\
        &= \frac{1}{n} \sum_{i=1}^n \mathbbm{1}_{(-\infty,b)}(x_i)- \frac{1}{n} \sum_{i=1}^n \mathbbm{1}_{(-\infty,a)}(x_i)\\
        &= F_n(b-)-F_n(a-)
        \end{align*}

        \item $ h_n((a,\infty)) = 1-F_n(a) $

    \begin{align*}
        h_n((a,\infty)) &= \frac{1}{n}\sum_{i=1}^n \mathbbm{1}_{(a,\infty)}(x_i)\\
        &= \lim_{t\rightarrow \infty}\frac{1}{n}\sum_{i=1}^n \mathbbm{1}_{(-\infty,t]}(x_i)- \mathbbm{1}_{(-\infty,a]}(x_i)\\
        &= \lim_{t\rightarrow \infty}F_n(t)-F_n(a)\\
        &= 1-F_n(a)
    \end{align*}
\end{enumerate}

  



\newpage

% Question 2
\section*{T2. Convergence of cumulative distribution functions}
A function $F : \mathbb{R} \rightarrow [0, 1]$ is called a cumulative distribution function, if $F$ is monotonically increasing and right continuous, and if we have $\lim_{t\rightarrow -\infty} F(t) = 0$ and $\lim_{t\rightarrow \infty} F(t) = 1$. Let $(F_n)_{n\in  \mathbb{N} }$ be a sequence of cumulative distribution functions, which converges uniformly to a function $F : \mathbb{R} \rightarrow \mathbb{R}$

\begin{enumerate}
\item  Give the definition of uniform convergence. Recall from your calculus lecture notes the following result: Given a sequence of continuous functions that converges uniformly, then the limit function is continuous. Recall the proof of that statement.
\item  Show that $F$ is a cumulative distribution function.
\item  Why is the latter result important for our approach to statistics?
\end{enumerate}

% Solution 2
\subsection*{Solution T2}

\begin{enumerate}
    \item We say that a sequence of functions ${f_n}$, defined on a common domain $A$, converges uniformly to a function $f$ on $A$, if for any $\epsilon > 0$, there exists a positive integer $N$ such that for all $n  \geq N$ and for all $x \in A$ we have $|f_n(x) - f(x)| < \epsilon$.

    \textbf{Given a sequence of continuous functions that converges uniformly,
    then the limit function is continuous.}

    \textit{Proof:}

    Let $\epsilon > 0$, $x \in A$.

    Uniform convergence implies that there exists a $N \in \mathbb{N}$ such that  $\forall x' \in A$ we have 
    \[
    |f_N(x') - f(x')| < \frac{\epsilon}{3}
    \]

    Since $f_N$ is continuous, $\exists \delta > 0$ such that  $\forall x' \in A$ with $|x'-a| < \delta$, $a \in A$ we have

    \[
    |f_N(x') - f_N(a)| < \frac{\epsilon}{3}
    \]

 Let $a \in A$ with $|x-a|<\delta|$. Then, by the triangle inequality, we have

    \begin{align*}
        |f(x)-f(a)| &\leq |f(x)-f_N(x)|+|f_N(x)-f_N(a)|+|f_N(a)-f(a)|\\
        &\leq \frac{\epsilon}{3}+\frac{\epsilon}{3}+\frac{\epsilon}{3}\\
        &\leq \epsilon
    \end{align*}


    \item We need to show that \( F \) is 
    \begin{enumerate}
        \item monotonically increasing,
        \item right continuous, and
        \item  \( \lim_{t\rightarrow -\infty} F(t) = 0 \) and \( \lim_{t\rightarrow \infty} F(t) = 1 \).
    \end{enumerate}
    \textit{Proof:}
    \begin{enumerate}
         
    
        \item \textit{Monotonicity:} 
        Let $a<b \implies \forall n \in \mathbb{N} \quad F_n(a) \leq F_n(b)$. 
        \[
        F(a)=\lim_{n \to \infty} F_n(a) \leq \lim_{n \to \infty} F_n(b)=F(b)    
        \]


        \item \textit{Right Continuity:} 
      
To establish right continuity of \( F \), we must show that for every \( t \in \mathbb{R} \),

\[
\lim_{s \downarrow t} F(s) = F(t).
\]


Since \( F_n \) is right continuous, we have 

\[
\lim_{s \downarrow t} F_n(s) = F_n(t).
\]

By taking the uniform limit as \( n \to \infty \), we get 

\[
\lim_{s \downarrow t} F(s) = \lim_{s \downarrow t} \lim_{n \to \infty} F_n(s) = \lim_{n \to \infty} \lim_{s \downarrow t} F_n(s) = \lim_{n \to \infty} F_n(t) = F(t).
\]


        \item \textit{Boundary Conditions:} The uniform convergence of \( F_n \) to \( F \) also guarantees that the boundary conditions at \(-\infty\) and \(+\infty\) will be preserved. Specifically, we consider the limits:
        \begin{align*}
            \lim_{t \to -\infty} F(t) &= \lim_{t \to -\infty} \lim_{n \to \infty} F_n(t) = \lim_{n \to \infty} \lim_{t \to -\infty} F_n(t) \\
            &= \lim_{n \to \infty} 0 = 0,
        \end{align*}
        and,
        \begin{align*}
            \lim_{t \to \infty} F(t) &= \lim_{t \to \infty} \lim_{n \to \infty} F_n(t) = \lim_{n \to \infty} \lim_{t \to \infty} F_n(t) \\
            &= \lim_{n \to \infty} 1 = 1.
        \end{align*}
        
        These steps are justified because uniform convergence allows us to switch the order of limits for functions, and \( F_n \) satisfy the CDF boundary conditions by definition.
        
    \end{enumerate}
    
    By confirming the monotonicity, right continuity, and boundary conditions, we have shown that \( F \) is a cumulative distribution function.

    \item \textit{Importance of Result for Statistics:}
    
    ECDF mostly converges uniformly and with the above results we know that the limit is also a CDF.

    So by doing experiments, we can learn something about the random mechanisms behind the experiments. This is because the ECDF converges to the true CDF for a sufficiently large number of samples (n).
    

\end{enumerate}





  


% Question 3
\section*{T3. Symmetric cumulative distribution function}
 We call a continuous cumulative distribution function $F$ symmetric in $c$, if $F(c+t)=1-F(c-t)$ for all $t \geq 0 $. $F$ is called \textit{symmetric}, if there is a $c$ such that $F$ is symmetric in $c$.

 \begin{enumerate}
    
 

    \item  Show that $F(c+t)=1-F(c-t)$ for all $t \geq 0$ implies the continuity of the cumulative distribution function $F$. (Why is it enough to prove left continuity?)

    \item  Assume that there is a continuous function $f: \mathbb{R} \rightarrow[0, \infty)$ such that

$$
F(t)=\int_{-\infty}^{t} f(x) d x \quad \forall t \in \mathbb{R}
$$

and

$$
\int_{-\infty}^{\infty} f(x) d x=1
$$

i.e. $F$ has density $f$.

\textit{Remark: Continuity of $f$ is not necessary for the definition. A necessary condition is measurability, a concept that is treated in measure theory.}

Show that $F$ is symmetric in $c$ if and only if $f(c+t)=f(c-t)$ for all $t \geq 0$.

\textit{Remark: This result stays true for piecewise continuous $f$.}

\item  For a continuous cumulative distribution function $F$, a median of $F$ is any real number $x$ such that $F(x)=1 / 2$. Why is a median an important parameter of a distribution? For any symmetric $F$, give an example of a median. May there be more than one median for a given $F$ ?

\item  Give a median of the normal distribution $\mathcal{N}\left(\mu, \sigma^{2}\right)$. Is it unique?
\end{enumerate}
% Solution 3
\subsection*{Solution T3}
  
\begin{enumerate}
    \item Show that $F(c+t)=1-F(c-t)$ for all $t \geq 0$ implies the continuity of the cumulative distribution function $F$. (Why is it enough to prove left continuity?)
    
As F is a CDF, it is right continuous. We need to show that it is also left continuous.
We have to show that for any $x \in \mathbb{R}$ and any sequence



\end{enumerate}



\end{document}